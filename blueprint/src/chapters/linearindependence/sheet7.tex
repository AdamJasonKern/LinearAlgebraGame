\section{Uniqueness Of Linear Combinations}

\begin{theorem}
  \label{theorem :  linear_combination_unique}
  \lean{LinearAlgebraGame.linear_combination_unique}
  \leanok
  \uses{definition : linear_independent, theorem : zero_smul}
  Let $S \subseteq V$ be a linearly independent set. If 
  $$\sum_{v \in T_1} a_v \cdot v = \sum_{v \in T_2} b_v \cdot v$$
  where $T_1, T_2$ are finite subsets of $S$, $a_v = 0$ for $v \notin T_1$, and $b_v = 0$ for $v \notin T_2$, then $a_v = b_v$ for all $v \in V$.
\end{theorem}

\begin{proof}
  This follows from the definition of linear independence: if a linear combination of linearly independent vectors equals zero, then all coefficients must be zero.
\end{proof}